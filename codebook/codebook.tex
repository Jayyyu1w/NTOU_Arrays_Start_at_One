\documentclass[10pt,twocolumn,oneside]{article}
\setlength{\columnsep}{10pt}                    %兩欄模式的間距
\setlength{\columnseprule}{0pt}                 %兩欄模式間格線粗細

\usepackage{amsthm}								%定義,例題
\usepackage{amssymb}
\usepackage{fontspec}							%設定字體
\usepackage{color}
\usepackage[x11names]{xcolor}
\usepackage{listings}							%顯示code用的
\usepackage{fancyhdr}							%設定頁首頁尾
\usepackage{graphicx}							%Graphic
\usepackage{enumerate}
\usepackage{titlesec}
\usepackage{amsmath}
\usepackage[CheckSingle, CJKmath]{xeCJK}
 \usepackage{CJKulem}

\usepackage{amsmath, courier, listings, fancyhdr, graphicx}
\topmargin=0pt
\headsep=5pt
\textheight=740pt
\footskip=0pt
\voffset=-50pt
\textwidth=545pt
\marginparsep=0pt
\marginparwidth=0pt
\marginparpush=0pt
\oddsidemargin=0pt
\evensidemargin=0pt
\hoffset=-42pt

%\renewcommand\listfigurename{圖目錄}
%\renewcommand\listtablename{表目錄}

%%%%%%%%%%%%%%%%%%%%%%%%%%%%%

\setmainfont{Consolas}
%\setmonofont{Ubuntu Mono}
\setmonofont{Consolas}
\setCJKmainfont{Noto Sans CJK TC}
\XeTeXlinebreaklocale "zh"						%中文自動換行
\XeTeXlinebreakskip = 0pt plus 1pt				%設定段落之間的距離
\setcounter{secnumdepth}{3}						%目錄顯示第三層

%%%%%%%%%%%%%%%%%%%%%%%%%%%%%
\makeatletter
\lst@CCPutMacro\lst@ProcessOther {"2D}{\lst@ttfamily{-{}}{-{}}}
\@empty\z@\@empty
\makeatother
\lstset{										% Code顯示
    language=C++,									% the language of the code
    basicstyle=\footnotesize\ttfamily, 					% the size of the fonts that are used for the code
    %numbers=left,									% where to put the line-numbers
    numberstyle=\footnotesize,					% the size of the fonts that are used for the line-numbers
    stepnumber=1,									% the step between two line-numbers. If it's 1, each line  will be numbered
    numbersep=5pt,									% how far the line-numbers are from the code
    backgroundcolor=\color{white},				% choose the background color. You must add \usepackage{color}
    showspaces=false,								% show spaces adding particular underscores
    showstringspaces=false,						% underline spaces within strings
    showtabs=false,								% show tabs within strings adding particular underscores
    frame=false,										% adds a frame around the code
    tabsize=2,										% sets default tabsize to 2 spaces
    captionpos=b,									% sets the caption-position to bottom
    breaklines=true,								% sets automatic line breaking
    breakatwhitespace=false,						% sets if automatic breaks should only happen at whitespace
    escapeinside={\%*}{*)},						% if you want to add a comment within your code
    morekeywords={*},								% if you want to add more keywords to the set
    keywordstyle=\bfseries\color{Blue1},
    commentstyle=\itshape\color{Red4},
    stringstyle=\itshape\color{Green4},
}


\begin{document}
\pagestyle{fancy}
\fancyfoot{}
%\fancyfoot[R]{\includegraphics[width=20pt]{ironwood.jpg}}
\fancyhead[C]{your school}
\fancyhead[L]{your team}
\fancyhead[R]{(\today) \thepage}
\renewcommand{\headrulewidth}{0.4pt}
\renewcommand{\contentsname}{Contents}

\scriptsize
\tableofcontents
\section{Brute Forcee}
\lstinputlisting{Brute Forcee/BF.cpp}

\section{DataStructure}

\subsection{SQRT}
\lstinputlisting{DataStructure/SQRT.cpp}

\subsection{Mo's Algorithm}
\lstinputlisting{DataStructure/Mo's Algorithm.cpp}

\section{Graph(Tree)}

\subsection{Eulerian Path and Circuit}
\lstinputlisting{Graph(Tree)/Eulerian Path & Circuit.cpp}

\subsection{Topological Sort}
\lstinputlisting{Graph(Tree)/Topological Sort.cpp}

\subsection{LCA}
\lstinputlisting{Graph(Tree)/LCA.cpp}

\subsection{樹上差分}
\lstinputlisting{Graph(Tree)/樹上差分.cpp}

\subsection{HLD with Segment tree}
\lstinputlisting{Graph(Tree)/heavy-light decomposition with Segment tree.cpp}

\subsection{區間修改Segment tree}
\lstinputlisting{Graph(Tree)/區間修改Segment tree.cpp}

\subsection{DSU on Tree}
\lstinputlisting{Graph(Tree)/dsu on tree.cpp}

\lstinputlisting{Other/funny.cpp}

\section{DP(Tree)}

\subsection{全點對距離Tree Distance}
\lstinputlisting{DP(Tree)/Tree Distance.cpp}

\subsection{最大獨立集Independent set}
\lstinputlisting{DP(Tree)/Independent set.cpp}

\subsection{最小點覆蓋Vertex Cover}
\lstinputlisting{DP(Tree)/Vertex Cover.cpp}

\subsection{最小支配集Dominating Set}
\lstinputlisting{DP(Tree)/Dominating Set.cpp}

\section{String}

\subsection{Trie}
\lstinputlisting{String/trie.cpp}

\subsection{01Trie}
\lstinputlisting{String/01trie.cpp}

\subsection{Hash}
\lstinputlisting{String/hash.cpp}



\end{document}
